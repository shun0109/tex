%%%%%%%%%%%%%%%%%%%%%%%%%%%%%%%%%%%%%%%%%
% Lachaise Assignment
% LaTeX Template
% Version 1.0 (26/6/2018)
%
% This template originates from:
% http://www.LaTeXTemplates.com
%
% Authors:
% Marion Lachaise & François Févotte
% Vel (vel@LaTeXTemplates.com)
%
% License:
% CC BY-NC-SA 3.0 (http://creativecommons.org/licenses/by-nc-sa/3.0/)
% 
%%%%%%%%%%%%%%%%%%%%%%%%%%%%%%%%%%%%%%%%%

%----------------------------------------------------------------------------------------
%	PACKAGES AND OTHER DOCUMENT CONFIGURATIONS
%----------------------------------------------------------------------------------------

\documentclass[fleqn]{article}
\setlength{\mathindent}{3pt}

%%%%%%%%%%%%%%%%%%%%%%%%%%%%%%%%%%%%%%%%%
% Lachaise Assignment
% Structure Specification File
% Version 1.0 (26/6/2018)
%
% This template originates from:
% http://www.LaTeXTemplates.com
%
% Authors:
% Marion Lachaise & François Févotte
% Vel (vel@LaTeXTemplates.com)
%
% License:
% CC BY-NC-SA 3.0 (http://creativecommons.org/licenses/by-nc-sa/3.0/)
% 
%%%%%%%%%%%%%%%%%%%%%%%%%%%%%%%%%%%%%%%%%

%----------------------------------------------------------------------------------------
%	PACKAGES AND OTHER DOCUMENT CONFIGURATIONS
%----------------------------------------------------------------------------------------

\usepackage{amsmath,amsfonts,stmaryrd,amssymb} % Math packages

\usepackage{enumerate} % Custom item numbers for enumerations

\usepackage[ruled]{algorithm2e} % Algorithms

\usepackage[framemethod=tikz]{mdframed} % Allows defining custom boxed/framed environments

\usepackage{listings} % File listings, with syntax highlighting
\lstset{
	basicstyle=\ttfamily, % Typeset listings in monospace font
}

%----------------------------------------------------------------------------------------
%	DOCUMENT MARGINS
%----------------------------------------------------------------------------------------

\usepackage{geometry} % Required for adjusting page dimensions and margins

\geometry{
	paper=a4paper, % Paper size, change to letterpaper for US letter size
	top=2.5cm, % Top margin
	bottom=3cm, % Bottom margin
	left=2.5cm, % Left margin
	right=2.5cm, % Right margin
	headheight=14pt, % Header height
	footskip=1.5cm, % Space from the bottom margin to the baseline of the footer
	headsep=1.2cm, % Space from the top margin to the baseline of the header
	%showframe, % Uncomment to show how the type block is set on the page
}

%----------------------------------------------------------------------------------------
%	FONTS
%----------------------------------------------------------------------------------------

\usepackage[utf8]{inputenc} % Required for inputting international characters
\usepackage[T1]{fontenc} % Output font encoding for international characters

\usepackage{XCharter} % Use the XCharter fonts

%----------------------------------------------------------------------------------------
%	COMMAND LINE ENVIRONMENT
%----------------------------------------------------------------------------------------

% Usage:
% \begin{commandline}
%	\begin{verbatim}
%		$ ls
%		
%		Applications	Desktop	...
%	\end{verbatim}
% \end{commandline}

\mdfdefinestyle{commandline}{
	leftmargin=10pt,
	rightmargin=10pt,
	innerleftmargin=15pt,
	middlelinecolor=black!50!white,
	middlelinewidth=2pt,
	frametitlerule=false,
	backgroundcolor=black!5!white,
	frametitle={Command Line},
	frametitlefont={\normalfont\sffamily\color{white}\hspace{-1em}},
	frametitlebackgroundcolor=black!50!white,
	nobreak,
}

% Define a custom environment for command-line snapshots
\newenvironment{commandline}{
	\medskip
	\begin{mdframed}[style=commandline]
}{
	\end{mdframed}
	\medskip
}

%----------------------------------------------------------------------------------------
%	FILE CONTENTS ENVIRONMENT
%----------------------------------------------------------------------------------------

% Usage:
% \begin{file}[optional filename, defaults to "File"]
%	File contents, for example, with a listings environment
% \end{file}

\mdfdefinestyle{file}{
	innertopmargin=1.6\baselineskip,
	innerbottommargin=0.8\baselineskip,
	topline=false, bottomline=false,
	leftline=false, rightline=false,
	leftmargin=2cm,
	rightmargin=2cm,
	singleextra={%
		\draw[fill=black!10!white](P)++(0,-1.2em)rectangle(P-|O);
		\node[anchor=north west]
		at(P-|O){\ttfamily\mdfilename};
		%
		\def\l{3em}
		\draw(O-|P)++(-\l,0)--++(\l,\l)--(P)--(P-|O)--(O)--cycle;
		\draw(O-|P)++(-\l,0)--++(0,\l)--++(\l,0);
	},
	nobreak,
}

% Define a custom environment for file contents
\newenvironment{file}[1][File]{ % Set the default filename to "File"
	\medskip
	\newcommand{\mdfilename}{#1}
	\begin{mdframed}[style=file]
}{
	\end{mdframed}
	\medskip
}

%----------------------------------------------------------------------------------------
%	NUMBERED QUESTIONS ENVIRONMENT
%----------------------------------------------------------------------------------------

% Usage:
% \begin{question}[optional title]
%	Question contents
% \end{question}

\mdfdefinestyle{question}{
	innertopmargin=1.2\baselineskip,
	innerbottommargin=0.8\baselineskip,
	roundcorner=5pt,
	nobreak,
	singleextra={%
		\draw(P-|O)node[xshift=1em,anchor=west,fill=white,draw,rounded corners=5pt]{%
		Question \theQuestion\questionTitle};
	},
}

\newcounter{Question} % Stores the current question number that gets iterated with each new question

% Define a custom environment for numbered questions
\newenvironment{question}[1][\unskip]{
	\bigskip
	\stepcounter{Question}
	\newcommand{\questionTitle}{~#1}
	\begin{mdframed}[style=question]
}{
	\end{mdframed}
	\medskip
}

%----------------------------------------------------------------------------------------
%	WARNING TEXT ENVIRONMENT
%----------------------------------------------------------------------------------------

% Usage:
% \begin{warn}[optional title, defaults to "Warning:"]
%	Contents
% \end{warn}

\mdfdefinestyle{warning}{
	topline=false, bottomline=false,
	leftline=false, rightline=false,
	nobreak,
	singleextra={%
		\draw(P-|O)++(-0.5em,0)node(tmp1){};
		\draw(P-|O)++(0.5em,0)node(tmp2){};
		\fill[black,rotate around={45:(P-|O)}](tmp1)rectangle(tmp2);
		\node at(P-|O){\color{white}\scriptsize\bf !};
		\draw[very thick](P-|O)++(0,-1em)--(O);%--(O-|P);
	}
}

% Define a custom environment for warning text
\newenvironment{warn}[1][Warning:]{ % Set the default warning to "Warning:"
	\medskip
	\begin{mdframed}[style=warning]
		\noindent{\textbf{#1}}
}{
	\end{mdframed}
}

%----------------------------------------------------------------------------------------
%	INFORMATION ENVIRONMENT
%----------------------------------------------------------------------------------------

% Usage:
% \begin{info}[optional title, defaults to "Info:"]
% 	contents
% 	\end{info}

\mdfdefinestyle{info}{%
	topline=false, bottomline=false,
	leftline=false, rightline=false,
	nobreak,
	singleextra={%
		\fill[black](P-|O)circle[radius=0.4em];
		\node at(P-|O){\color{white}\scriptsize\bf i};
		\draw[very thick](P-|O)++(0,-0.8em)--(O);%--(O-|P);
	}
}

% Define a custom environment for information
\newenvironment{info}[1][Info:]{ % Set the default title to "Info:"
	\medskip
	\begin{mdframed}[style=info]
		\noindent{\textbf{#1}}
}{
	\end{mdframed}
}
 % Include the file specifying the document structure and custom commands

%----------------------------------------------------------------------------------------
%	ASSIGNMENT INFORMATION
%----------------------------------------------------------------------------------------

\title{Beta Function} % Title of the assignment

%\author{Shun Miyamoto\\ \texttt{smiyamot@bu.edu}} % Author name and email address

%\date{Boston University --- \today} % University, school and/or department name(s) and a date

%----------------------------------------------------------------------------------------

\begin{document}

\maketitle % Print the title

%----------------------------------------------------------------------------------------
%	INTRODUCTION
%----------------------------------------------------------------------------------------


%----------------------------------------------------------------------------------------
%	PROBLEM 1
%----------------------------------------------------------------------------------------

\section{Beta Functions for $\phi ^{4}$ theory} % Numbered section
\subsection{Setting}
Let us start with the bare Lagrangian,
\begin{align}
\mathcal{L} &= \frac{1}{2}\partial_\mu \phi \partial^{\mu} \phi - \frac{1}{2}m_0^{2}\phi ^{2} - \frac{1}{4!} \lambda_0 \phi^{4}
\end{align}
Then, with the renormalized field,
\begin{align}
\phi &= Z^{1/2} \phi _r
\end{align}
the renormalized Lagrangian takes tthe form;
\begin{align}
\mathcal{L} &= \mathcal{L}_r + \Delta \mathcal{L} \nonumber  &\\
&= (\frac{1}{2}\partial _\mu \phi_r \partial ^{\mu} \phi_r - \frac{1}{2}m^{2}\phi_r ^{2} - \frac{1}{4!}\lambda \phi_r ^{4}) + (\frac{1}{2} \Delta_Z \partial_\mu \phi_r \partial^{\mu} \phi_r- \frac{1}{2}\Delta_m \phi_r ^{2} - \frac{1}{4!}\Delta _{\lambda} \phi_r)&
\end{align}
where the counter terms are denoted as;
\begin{align}
\Delta_Z =& Z-1, \Delta_m = m_0 ^{2} Z -m^{2}, \Delta _{\lambda} = \lambda_0 Z^{2} - \lambda&
\end{align}
Also consider the renormalized Green's functions defined with the renormalized fields and depend on the scale $\mu$, the coupling $\lambda$ and the mass $m^{2}$;
\begin{align}
\textit{G}_r(x_1,x_2,...,x_n:\mu, \lambda, m^{2} ) &= \langle \phi_r(x_1) ... \phi_r(x_n) \rangle&
\end{align}
On the other hand, the bare Green's functions are defined with the bare fields and depend on the bare coupling $\lambda _0$ and the bare mass $m_0^{2}$:
\begin{align}
\textit{G}(x_1,x_2,...,x_n: \lambda_0, m_0^{2} ) &= \langle \phi(x_1) ... \phi(x_n) \rangle&
\end{align}
Recalling that $\phi(x) = Z^{1/2} \phi_r (x)$, where Z is depending on the scaling $\mu$, one may deduce that:
\begin{align}
\textit{G}(x_1,x_2,...,x_n: \lambda_0, m_0^{2} ) &=  Z^{1/2} \textit{G}_r(x_1,x_2,...,x_n:\mu, \lambda, m^{2} ) &
\end{align}
under the infinitesimal transformations $\mu \rightarrow \mu + \delta \mu$, $\lambda \rightarrow \lambda +\delta \lambda$, $m^{2} \rightarrow m^{2} + \delta m^{2}$ combined with $Z \rightarrow Z + \delta Z$:
\begin{align}
(\delta \mu \frac{\partial}{\partial \mu} + \delta \lambda \frac{\partial}{\partial \lambda }+ \delta Z \frac{\partial}{\partial Z} + \delta m^{2} \frac{\partial}{\partial m^{2}})Z^{n/2} \textit{G}_r(x_1,x_2,...,x_n:\mu, \lambda, m^{2} ) & = 0 &\nonumber \\
\Leftrightarrow (\delta \mu \frac{\partial}{\partial \mu} + \delta \lambda \frac{\partial}{\partial \lambda }+\frac{n}{2}\frac{\delta Z}{Z} + \delta m^{2} \frac{\partial}{\partial m^{2}})\textit{G}_r(x_1,x_2,...,x_n:\mu, \lambda, m^{2} ) & = 0 & 
\end{align}
Eventually, multiplying by $\frac{\mu}{\delta \mu}$, and rewriting the differentials as derivatives taken at constant value for the bare coupling, it is in the form:
\begin{align}
(\mu \frac{\partial}{\partial \mu} + \beta \frac{\partial}{\partial \lambda} + n\gamma + \beta_m m^{2} \frac{\partial}{\partial m^{2}}) \textit{G}_r &=0&
\end{align}
where:
\begin{align}
\beta &= \mu \frac{\partial \lambda}{\partial \mu}
 = \frac{\partial \lambda}{\partial \ln \mu}& \\
 \gamma & = \frac{1}{2}\frac{\mu}{Z} \frac{\partial Z}{\partial \mu} = \frac{\partial \ln Z}{\partial \ln \mu}& \\
 \beta_m &= \frac{\mu}{m^{2}}\frac{\partial m^{2}}{\partial \mu} = \frac{\partial \ln m^{2}}{\partial \ln \mu}&
\end{align}




\subsection{Computation}
First, let us define the renormalization conditions as following:
\begin{align}
&\textit{G}_r ^{(2)}( p^{2} = m^{2}) = \frac{i}{p^{2} -m^{2}} + (reg.)&\\
&i\ \mathcal{A}(s=4m^{2}, t= u =0) = -i\lambda&
\end{align}
where $G_r ^{2}$ is a connected 2-point function, and $i\mathcal{A}$ stands for the amplitude for the scattering two scalars into two scalars. Also, the Mandelstam variables are $s = (p_1 + p_2)^{2}$, $t= (p_3 -p_1)^{2}$, $u =(p_4 - p_1)^{2}$.\\
Start with the 4-point function on 4-dimension and up to 1-loop correction. The 1-loop contribution to s-channel is given by:
\begin{align}
\Sigma (p^{2}) &= \frac{(-i\lambda)^{2}}{2}\int\frac{d^{4}k}{(2\pi)^{4}}\frac{i}{k^{2}-m^{2}+i\epsilon}\frac{i}{(p-k)^{2}-m^{2} + i\epsilon}&
\end{align}
where $p = p_1 + p_2 = p_3 + p_4$ is the external momentum. Suppose that one analytically continues to a space of $d-1$ spatial and $1$ time dimensions, then the momentum in the integral can be written in;
\begin{align}
k^{\mu} = (k_0, k_1,...,k_{d-1})
\end{align}
But the external momenta are:
\begin{align}
p^{\mu} = (p_0,p_1,p_2,p_4,0,...,0)
\end{align}
in $d$-dimensional Minkowski space.\\
With this note, one may have a $d$-dimensional integral:
\begin{align}
I = \int \frac{d^{d}k}{(2\pi)^d}\mu^{4-d} \frac{1}{k^{2}-m^{2}+i\epsilon} \frac{1}{(p-k)^{2}-m^{2} + i\epsilon}
\end{align}
Here, the scale $\mu$ is introduced to  compensate the change in the dimensions of fields and couplings. Notice that this integral is convergent for $d<4$. In oder to continue, it is useful to apply the Feynman parametrization:
\begin{align}
\frac{1}{AB} = \int_{0}^{1}dx \frac{1}{[xA+(1-x)B]^{2}}
\end{align}
Then with writing $\epsilon \equiv 4-d$, the integral would be:
\begin{align}
I &= \int \frac{d^{d}k}{(2\pi)^d} \int_{0}^{1}dx \mu^{\epsilon} \frac{1}{[x((p-k)^{2}-m^{2}) + (1-x)(k^{2}-m^{2})]^{2}} &\nonumber\\
&= \int \frac{d^{d}k}{(2\pi)^d} \int_{0}^{1}dx \mu^{\epsilon} \frac{1}{[k^{2} -2xk\cdot p + xp^{2} -m^{2}]^{2}}& \nonumber \\
&= \int \frac{d^{d}k}{(2\pi)^d} \int_{0}^{1}dx \mu^{\epsilon} \frac{1}{[(k-xp)^{2}+x(1-x)p^{2}-m^{2}]^{2}}& \nonumber \\
&=  \int_{0}^{1}dx\int \frac{d^{d}l}{(2\pi)^d} \mu^{\epsilon} \frac{1}{[l^{2}-\Delta^{2} +i\epsilon]^{2}}& \text{($l^{\mu} \equiv k^{\mu}-xp^{\mu}$, $\Delta^{2} \equiv m^{2} - x(1-x)p^{2}$}) \nonumber   \\
& = i\mu^{\epsilon} \int_{0}^{1} dx \int \frac{d^{d}l_E}{(2\pi)^d} \mu^{\epsilon} \frac{1}{[l_{E}^{2}-\Delta^{2} +i\epsilon]^{2}}&  \text{(Wick rotated)} \nonumber \\
&=  \frac{1}{(2\pi)^{d}}i\mu^{\epsilon} \int_{0}^{1} dx \int d \Omega _d \int dl_E \frac{l_{E}^{d-1}}{[l_{E}^{2}-\Delta^{2} +i\epsilon]^{2}}&
\end{align}
But notice the Gaussian integral is:
\begin{align}
\int_{-\infty}^{\infty}dx e^{-x^{2}} = \sqrt{\pi}
\end{align}
Then,
\begin{align}
(\sqrt{\pi})^{d} &= (\int_{-\infty}^{\infty}dx e^{-x^{2}})^{d} = \int d^{d} x e^{-\Sigma _{i=1} ^{d} x_{i} ^{2}}& \nonumber \\ 
& =\int d\Omega _{d} \int dx x^{d-1}e^{-x^{2}}& \nonumber \\ 
& = \int d\Omega _{d} \int _{0}^{\infty} \frac{dx^{2}}{2} (x^2)^{d/2 - 1} e^{-x^2}& \nonumber  \\
& = \int d \Omega _{d} \frac{1}{2} \Gamma (d/2) &
\end{align}
Thus, the integral being discussed is:
\begin{align}
I &= i \mu ^{\epsilon} \int_{0}^{1} dx \frac{2\pi ^{d/2}}{\Gamma(d/2)} \frac{1}{(2\pi)^d} \int _{0}^{\infty}dl_{E} \frac{l_{E}^{d-1}}{[l_{E}^{2} -\Delta ^{2} +i\epsilon]^{2}} & \nonumber \\
&= \frac{i\mu^{\epsilon} \pi ^{d/2}}{(2\pi)^{d}\Gamma(d/2)} \int _{0}^{1} dx  \int _{0}^{\infty}dl_{E}^{2} \frac{(l_{E}^{2})^{d/2 -1}}{[l_{E}^{2} -\Delta ^{2} +i\epsilon]^{2}}&\nonumber \\
&= \frac{i\mu^{\epsilon} \pi ^{d/2}}{(2\pi)^{d}\Gamma(d/2)} \int _{0}^{1} dx \frac{1}{[\Delta ^{2} -i\epsilon]^{2-d/2}} \frac{\Gamma(d/2)\Gamma(2-d/2)}{\Gamma(2)}& \nonumber \\
&= \frac{i\mu^{\epsilon} \pi ^{d/2}}{(2\pi)^{d}} \Gamma(2-\frac{d}{2})\int_{0} ^{1} dx\frac{1}{(\Delta ^{2})^{2-d/2}}& \nonumber \\
&= \frac{i\mu^{\epsilon} \pi ^{d/2}}{(2\pi)^{d}} \Gamma(\frac{\epsilon}{2})\int_{0} ^{1} dx\frac{1}{(\Delta ^{2})^{\epsilon/2}}&
\end{align}
On the third line, the relation between the beta and gamma functions was used:
\begin{align}
\int _{0}^{\infty} dt \frac{t^{m-1}}{(t^{2}+\Delta^{2})^{n}} = \frac{1}{(\Delta^{2})^{n-m}}\frac{\Gamma(m)\Gamma(n-m)}{\Gamma(n)}
\end{align}
Notice that $\Gamma(z)$ has poles at $z \in \mathbb{Z}_{\leq 0}$. Therefore, the integral I has poles for $d= 4,6,8,\cdots$. As mentioned above, our particular interest is in the limit $d \rightarrow 4$, that is, $\epsilon \rightarrow 0$. Then one may have expansions:
\begin{align}
\Gamma(\frac{\epsilon}{2}) &\simeq \frac{2}{\epsilon} - \gamma_{E} - \mathcal{O(\epsilon)} \\
\frac{1}{(\Delta^{2})^{\epsilon /2}} &\simeq 1 -\frac{\epsilon}{2} \ln \Delta ^{2} + \mathcal{O}(\epsilon ^{2})\\
\mu^{\epsilon} = (\mu^{2})^{\epsilon /2} &\simeq 1 + \frac{2}{\epsilon}\ln \mu^{2} + \cdots
\end{align}
where $\gamma_{E}$ is the Eüler-Mascheroni constant. And therefore,
\begin{align}
\mu^{\epsilon}\Gamma(\frac{\epsilon}{2})\frac{1}{(\Delta^{2})^{\epsilon /2}} \simeq \frac{2}{\epsilon} - \gamma_{E} - \ln \frac{\Delta ^{2}}{\mu^{2}} + \mathcal{O}(\epsilon)
\end{align}
This reads:
\begin{align}
I = \frac{i}{16\pi ^{2}} \int_{0}^{1} dx (\frac{2}{\epsilon} - \gamma_{E} - \ln \frac{m^{2} - x(1-x)p^{2}}{\mu^{2}} + \mathcal{O}(\epsilon))
\end{align}
The contribution is:
\begin{align}
\Sigma(p^{2}) &= -\frac{(-i\lambda)^{2}}{2}I \nonumber \\
&= \frac{i\lambda^{2}}{32\pi ^{2}} \int_{0}^{1} dx (\frac{2}{\epsilon} - \gamma_{E} - \ln \frac{m^{2} - x(1-x)p^{2}}{\mu^{2}} + \mathcal{O}(\epsilon))
\end{align}
According to the renormalization condition:
\begin{align}
i\mathcal{A}(s= 4m^{2}, t=u=0) &= -i\lambda + \Sigma(4m^{2}) + 2\Sigma(0) - i\delta_{\lambda} \nonumber \\
&= -i\lambda \nonumber \\
\Leftrightarrow \Sigma(4m^{2}) + 2\Sigma(0) -i\delta_{\lambda} &= 0 \nonumber \\
\Leftrightarrow \delta _{\lambda} &= \frac{\lambda^{2}}{32\pi^{2}} \int_{0}^{1} dx (\frac{6}{\epsilon} - 3\gamma_{E} -3 \ln \frac{m^{2}}{\mu^{2}} - \ln 4(1-x(1-x))) \nonumber \\
&= \frac{3\lambda^{2}}{32\pi^{2}} \ln\frac{\zeta}{\mu^{2}} + \text{finite term}
\end{align}
Similarly for the case of the two-point function. Define $-iM^{2}(p^{2})$ as the sum of all one-particle irreducible insertions (1PI) into the propagator, the full propagator is:
\begin{align}
\Delta_{F} = \frac{i}{p^{2} - m^{2} -M^{2}(p^{2})}
\end{align}
The renormalized condition imposed on this propagator requires that $p^{2} = m^{2}$ should be the pole, and the residue is supposed to be 1. Expanding $M^{2}$ about $p^{2} = m^{2}$, one may have:
\begin{align}
\Delta_{F} = \frac{i}{p^{2} - m^{2} - (M^{2}(m^{2}) + (p^{2} - m^{2})\frac{d}{dp^{2}}M^{2}(m^{2}) + \cdots)}
\end{align}
Therefore, the renormalized condition can be reinterpreted into two conditions;
\begin{align}
M^{2}(p^{2})|_{p^{2} = m^{2}} = 0 \text{       and      } \frac{d}{dp^{2}}M^{2}(p^{2})|_{p^{2} = m^{2}} = 0
\end{align}
Then, explicitly, again, to 1-loop order:
\begin{align}
-iM^{2}(p^{2}) &= -\frac{i\lambda}{2}\int\frac{d^{4}k}{(2\pi)^{4}}\frac{i}{k^{2}-m^{2}}+ i(p^{2} \delta_{z} - \delta_{m})
\end{align}
With the same method, regularizing the integral:
\begin{align}
J&= -\frac{i\lambda}{2}\mu^{\epsilon}\int\frac{d^{d}k}{(2\pi)^{d}} \frac{i}{k^{2} - m^{2}} & \nonumber \\
&= \frac{-i\lambda}{2}\frac{\mu^{\epsilon}}{(2\pi)^{d}} \int d\Omega_{d} \int_{0}^{\infty} dk_{E}\frac{k_{E}^{d-1}}{k_{E}^{2} + m^{2} -i\epsilon}& \nonumber \\
& =  \frac{-i\lambda}{2}\frac{\mu^{\epsilon}}{(2\pi)^{d}} \frac{\pi^{d/2}}{\Gamma(d/2)} \int_{0}^{\infty} dk_{E}^{2} \frac{(k_{E}^{2})^{d/2 -1}}{k_{E}^{2} + m^{2} -i\epsilon} \nonumber \\
& = \frac{-i\lambda}{2}\frac{\mu^{\epsilon}}{(2\pi)^{d}}\frac{\pi^{d/2}}{\Gamma(d/2)} \frac{1}{(m^{2})^{1-d/2}} \frac{\Gamma(\frac{d}{2})\Gamma(1-d/2)}{\Gamma(1)} \nonumber \\
& = \frac{-i\lambda}{2}\frac{\mu^{\epsilon}}{(2\pi)^{d}}\pi^{d/2} \frac{1}{(m^{2})^{1-d/2}}\Gamma(1-d/2)
\end{align}
Using the property of Gamma function:
\begin{align}
z\Gamma(z) = \Gamma(z+1)
\end{align}
one may rewrite:
\begin{align}
(1-d/2)\Gamma(1-d/2) &= \Gamma(2-d/2) = \Gamma(\epsilon / 2) \nonumber \\
\Leftrightarrow \Gamma(1-d/2) &= \frac{\Gamma(\epsilon/2)}{1-d/2}
\end{align}
Also, notice that:
\begin{align}
(m^{2})^{1-d/2} = (m^{2})^{-1}(m^{2})^{\epsilon/2}
\end{align}
Finally, the expansion around the limit $d \rightarrow 4$ is given as following:
\begin{align}
J &= \frac{-i\lambda}{2}\frac{\pi^{d/2}}{(2\pi)^{d}}\frac{m^{2}}{1-d/2}\frac{\mu^{\epsilon}}{(m^2)^{\epsilon/2}} \Gamma(\epsilon/2) \nonumber \\
&\overrightarrow {\scalebox{0.7}{$d \to 4$}} \frac{i\lambda}{32\pi^{2}}m^{2} (\frac{2}{\epsilon} - \gamma_{E} - \ln \frac{m^{2}}{\mu^{2}} + \mathcal{O}(\epsilon)) \nonumber \\
& = \frac{i\lambda}{32\pi^{2}}m^{2}\ln\frac{\zeta}{\mu^{2}} + \text{finite terms.}
\end{align}
Therefore, comparing with the renormalization condition redefined above, 
\begin{align}
\delta_{z} &= 0 \\
\delta_{m^{2}} &= \frac{\lambda}{32\pi^{2}} m^{2}\ln\frac{\zeta}{\mu^{2}}
\end{align}
The computations of the renormalization group functions are carried with the definitions of the counter-terms $\delta_Z = Z-1$, $\delta_{\lambda} = \lambda_{0}Z^{2} - \lambda$, and $\delta_{m^{2}} = m_{0}^{2}Z - m^{2}$:
\begin{align}
\gamma&= \frac{1}{2}\frac{\mu}{Z}\frac{\partial Z}{\partial \mu} = \frac{1}{2}\mu \frac{\partial \delta_{Z}}{\partial \mu} = 0\\
\beta & = \mu \frac{\partial \lambda}{\partial \mu} = \mu \frac{\partial}{\partial \mu}(\lambda_{0} + 2\delta_{Z}\lambda_{0} - \delta_{\lambda} + (\delta_{Z})^{2} \lambda_{0}) \nonumber \\
&= -\mu \frac{\partial \delta_{\lambda}}{\partial \mu} = \frac{3\lambda ^{2}}{16\pi^{2}}\\
\beta_{m}&= \frac{\mu}{m^{2}}\frac{\partial m^{2}}{\partial \mu} = -\frac{\mu}{m^{2}}\frac{\partial \delta_{m^{2}}}{\partial \mu} \nonumber \\
&= \frac{\lambda}{16\pi^{2}}
\end{align}
Solving for $\lambda$ and $m^{2}$:
\begin{align}
\ln \frac{\mu}{\mu_{0}} & = \frac{16\pi^{2}}{3}(\frac{1}{\lambda_{0}} - \frac{1}{\lambda}) \nonumber \\
\Leftrightarrow \lambda(\mu) &= \frac{\lambda_{0}}{1-\frac{3\lambda_{0}}{16\pi^{2}}\ln \frac{\mu}{\mu_{0}}}\\
\ln \frac{m^{2}}{m_{0}^{2}} & = \frac{1}{3}\ln \frac{\lambda}{\lambda_{0}} \nonumber \\
\Leftrightarrow m^{2}(\mu) &= [1-\frac{3\lambda_{0}}{16\pi^{2}} \ln \frac{\mu}{\mu_{0}}]^{-1/3} m_{0}^{2}
\end{align}



%----------------------------------------------------------------------------------------
%	PROBLEM 2
%----------------------------------------------------------------------------------------

%----------------------------------------------------------------------------------------

\clearpage
\appendix
\section{Feynman Parametrization}
It is known the Feynman parametrization in general form:
\begin{align}
\frac{1}{a_{1}a_{2}\cdots a_{n}} = \int _0^1 dx_{1} \cdots \int_0^1 dx_{n} \frac{\Gamma(n) \delta(1-x_{1}- \cdots -x_{n})}{(a_{1}x_{1} + \cdots + a_{n}x_{n})^{n}}
\end{align}
Let us prove it briefly.
\begin{align}
I&= \frac{1}{a_{1} \cdots a_{n}} \nonumber \\
&= \int _{0}^{\infty} dt_{1} \cdots \int _{0}^{\infty} dt_{n} e^{-(a_{1}t_{1} + \cdots + a_{n} t_{n})} \nonumber \\
&= \int _{0}^{\infty} dt \int _{0}^{\infty} dt_{1} \cdots \int _{0}^{\infty} dt_{n} \delta (t- t_{1} - \cdots -t_{n}) e^{-(a_{1}t_{1} + \cdots + a_{n} t_{n})} \nonumber 
\end{align}
Substituting $t_{i} = tx_{i}$,
\begin{align}
I &= \int _{0}^{\infty} dt \int _{0}^{\infty}dx_{1} \cdots \int _{0}^{\infty}dx_{n} t^{n}\delta(t(1-x_{1} - \cdots -t_{n})) e^{-t(a_{1}x_{1} + \cdots + a_{n}x_{n})} \nonumber \\
&= \int _{0}^{\infty}dx_{1} \cdots \int _{0}^{\infty}dx_{n} \delta (1-x_{1} - \cdots -x_{n}) \int_{0} ^{\infty} dt t^{n-1}e^{-t(a_{1}x_{1} + \cdots + a_{n}x_{n})} \nonumber \\
&= \int _{0}^{1} dx_{1} \cdots \int_{0}^{1} dx_{n}\frac{\Gamma(n) \delta (1-x_{1} - \cdots - x_{n})}{(a_{1}x_{1}+ \cdots + a_{n}x_{n})}
\end{align}
%\section{Second}

\end{document}
