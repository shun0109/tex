\documentclass[fleqn]{article}
\setlength{\mathindent}{3pt}
\usepackage{bbold}
\usepackage{hyperref}
\usepackage[overload]{empheq}
%%%%%%%%%%%%%%%%%%%%%%%%%%%%%%%%%%%%%%%%%
% Lachaise Assignment
% Structure Specification File
% Version 1.0 (26/6/2018)
%
% This template originates from:
% http://www.LaTeXTemplates.com
%
% Authors:
% Marion Lachaise & François Févotte
% Vel (vel@LaTeXTemplates.com)
%
% License:
% CC BY-NC-SA 3.0 (http://creativecommons.org/licenses/by-nc-sa/3.0/)
% 
%%%%%%%%%%%%%%%%%%%%%%%%%%%%%%%%%%%%%%%%%

%----------------------------------------------------------------------------------------
%	PACKAGES AND OTHER DOCUMENT CONFIGURATIONS
%----------------------------------------------------------------------------------------

\usepackage{amsmath,amsfonts,stmaryrd,amssymb} % Math packages

\usepackage{enumerate} % Custom item numbers for enumerations

\usepackage[ruled]{algorithm2e} % Algorithms

\usepackage[framemethod=tikz]{mdframed} % Allows defining custom boxed/framed environments

\usepackage{listings} % File listings, with syntax highlighting
\lstset{
	basicstyle=\ttfamily, % Typeset listings in monospace font
}

%----------------------------------------------------------------------------------------
%	DOCUMENT MARGINS
%----------------------------------------------------------------------------------------

\usepackage{geometry} % Required for adjusting page dimensions and margins

\geometry{
	paper=a4paper, % Paper size, change to letterpaper for US letter size
	top=2.5cm, % Top margin
	bottom=3cm, % Bottom margin
	left=2.5cm, % Left margin
	right=2.5cm, % Right margin
	headheight=14pt, % Header height
	footskip=1.5cm, % Space from the bottom margin to the baseline of the footer
	headsep=1.2cm, % Space from the top margin to the baseline of the header
	%showframe, % Uncomment to show how the type block is set on the page
}

%----------------------------------------------------------------------------------------
%	FONTS
%----------------------------------------------------------------------------------------

\usepackage[utf8]{inputenc} % Required for inputting international characters
\usepackage[T1]{fontenc} % Output font encoding for international characters

\usepackage{XCharter} % Use the XCharter fonts

%----------------------------------------------------------------------------------------
%	COMMAND LINE ENVIRONMENT
%----------------------------------------------------------------------------------------

% Usage:
% \begin{commandline}
%	\begin{verbatim}
%		$ ls
%		
%		Applications	Desktop	...
%	\end{verbatim}
% \end{commandline}

\mdfdefinestyle{commandline}{
	leftmargin=10pt,
	rightmargin=10pt,
	innerleftmargin=15pt,
	middlelinecolor=black!50!white,
	middlelinewidth=2pt,
	frametitlerule=false,
	backgroundcolor=black!5!white,
	frametitle={Command Line},
	frametitlefont={\normalfont\sffamily\color{white}\hspace{-1em}},
	frametitlebackgroundcolor=black!50!white,
	nobreak,
}

% Define a custom environment for command-line snapshots
\newenvironment{commandline}{
	\medskip
	\begin{mdframed}[style=commandline]
}{
	\end{mdframed}
	\medskip
}

%----------------------------------------------------------------------------------------
%	FILE CONTENTS ENVIRONMENT
%----------------------------------------------------------------------------------------

% Usage:
% \begin{file}[optional filename, defaults to "File"]
%	File contents, for example, with a listings environment
% \end{file}

\mdfdefinestyle{file}{
	innertopmargin=1.6\baselineskip,
	innerbottommargin=0.8\baselineskip,
	topline=false, bottomline=false,
	leftline=false, rightline=false,
	leftmargin=2cm,
	rightmargin=2cm,
	singleextra={%
		\draw[fill=black!10!white](P)++(0,-1.2em)rectangle(P-|O);
		\node[anchor=north west]
		at(P-|O){\ttfamily\mdfilename};
		%
		\def\l{3em}
		\draw(O-|P)++(-\l,0)--++(\l,\l)--(P)--(P-|O)--(O)--cycle;
		\draw(O-|P)++(-\l,0)--++(0,\l)--++(\l,0);
	},
	nobreak,
}

% Define a custom environment for file contents
\newenvironment{file}[1][File]{ % Set the default filename to "File"
	\medskip
	\newcommand{\mdfilename}{#1}
	\begin{mdframed}[style=file]
}{
	\end{mdframed}
	\medskip
}

%----------------------------------------------------------------------------------------
%	NUMBERED QUESTIONS ENVIRONMENT
%----------------------------------------------------------------------------------------

% Usage:
% \begin{question}[optional title]
%	Question contents
% \end{question}

\mdfdefinestyle{question}{
	innertopmargin=1.2\baselineskip,
	innerbottommargin=0.8\baselineskip,
	roundcorner=5pt,
	nobreak,
	singleextra={%
		\draw(P-|O)node[xshift=1em,anchor=west,fill=white,draw,rounded corners=5pt]{%
		Question \theQuestion\questionTitle};
	},
}

\newcounter{Question} % Stores the current question number that gets iterated with each new question

% Define a custom environment for numbered questions
\newenvironment{question}[1][\unskip]{
	\bigskip
	\stepcounter{Question}
	\newcommand{\questionTitle}{~#1}
	\begin{mdframed}[style=question]
}{
	\end{mdframed}
	\medskip
}

%----------------------------------------------------------------------------------------
%	WARNING TEXT ENVIRONMENT
%----------------------------------------------------------------------------------------

% Usage:
% \begin{warn}[optional title, defaults to "Warning:"]
%	Contents
% \end{warn}

\mdfdefinestyle{warning}{
	topline=false, bottomline=false,
	leftline=false, rightline=false,
	nobreak,
	singleextra={%
		\draw(P-|O)++(-0.5em,0)node(tmp1){};
		\draw(P-|O)++(0.5em,0)node(tmp2){};
		\fill[black,rotate around={45:(P-|O)}](tmp1)rectangle(tmp2);
		\node at(P-|O){\color{white}\scriptsize\bf !};
		\draw[very thick](P-|O)++(0,-1em)--(O);%--(O-|P);
	}
}

% Define a custom environment for warning text
\newenvironment{warn}[1][Warning:]{ % Set the default warning to "Warning:"
	\medskip
	\begin{mdframed}[style=warning]
		\noindent{\textbf{#1}}
}{
	\end{mdframed}
}

%----------------------------------------------------------------------------------------
%	INFORMATION ENVIRONMENT
%----------------------------------------------------------------------------------------

% Usage:
% \begin{info}[optional title, defaults to "Info:"]
% 	contents
% 	\end{info}

\mdfdefinestyle{info}{%
	topline=false, bottomline=false,
	leftline=false, rightline=false,
	nobreak,
	singleextra={%
		\fill[black](P-|O)circle[radius=0.4em];
		\node at(P-|O){\color{white}\scriptsize\bf i};
		\draw[very thick](P-|O)++(0,-0.8em)--(O);%--(O-|P);
	}
}

% Define a custom environment for information
\newenvironment{info}[1][Info:]{ % Set the default title to "Info:"
	\medskip
	\begin{mdframed}[style=info]
		\noindent{\textbf{#1}}
}{
	\end{mdframed}
}



\begin{document}
Start with some corrections for previous note. Instead of $\frac{1}{\varphi_{2} ^{2}}$, putting a scale into a metric as $\frac{k}{\varphi _{2} ^{2}}$. The geodesics equations for $\dot \varphi_{1} =0$ leads $\dot \varphi_{2} / \varphi _{2} = c$ with some real constant $c$. Integration with respective to s yields $\varphi _{2} = \varphi _{2} ^{0}e^{cs}$. Considering the condition:
\begin{align}
1=  G_{ij} \dot \varphi_{i} \dot \varphi _{j} = \frac{k}{\varphi_{2} ^{2}} (\dot \varphi _{1} ^{2} + \dot \varphi _{2} ^{2}) = k c^{2} 
\end{align}
Hence, $c= \pm \frac{1}{\sqrt{k}}$, and the vertical geodesics are $(\varphi _{1} , \varphi _{2}) = (\varphi _{1} ^{0}, \varphi _{2} ^{0} e^{\pm s / \sqrt{k}})$ with signs on the exponential standing for the direction of lines traverse. Let us observe how these geodesics alter their form as the metric flows. As previously seen, at the scale $\mu = \Lambda (1-\epsilon)$, where $\epsilon \ll 1$, $\varphi _{1}$ no longer remains constant. From one of the geodesics equations, it can be deduced   that:
\begin{align}
&\dot \varphi _{1} = \frac{\epsilon}{8\pi ^{2} k} \lbrace (M_{2} - M_{1} )\dot \varphi _{1} \mp \frac{2m}{\sqrt{k}} \varphi _{2} \rbrace \\
\Rightarrow & \dot \varphi_{1} = \mp \frac{\epsilon}{4\pi^{2} k} \frac{m}{\sqrt{k}}\varphi_{2} + O(\epsilon^{2})
\end{align}
from which:
\begin{align}
\label{eq:8}
\varphi_{1} = \varphi _{1} ^{0} \mp \frac{\epsilon}{4\pi ^{2} k} m (\varphi _{2} - \varphi_{2} ^{0})
\end{align}
Also, the Hessian matrix needs to be fixed. Since the potential with the interaction terms is $V = \frac{1}{2} M_{ij} \varphi _{i}\varphi_{j} + \frac{\lambda} {4} (\varphi_{1}^{2} + \varphi_{2} ^{2} )^{2}$, the Hessian is given:
\begin{align}
H = 
\begin{bmatrix}
M_{1} + \lambda(3\varphi_{1}^{2} + \varphi_{2} ^{2}) & m + 2\lambda \varphi_{1}\varphi_{2} \\
m + 2\lambda \varphi_{1}\varphi_{2} & M_{2} + \lambda(3\varphi_{2}^{2} + \varphi_{1} ^{2})
\end{bmatrix}
\end{align}
Therefore, the flow equations for the masses are:
\begin{align}[left=\empheqlbrace]
\label{eq:1}
\mu \frac{\partial}{\partial \mu} M_{1} &= \frac{\lambda}{16\pi^{2}} (-4\mu^{2} + 3M_{1} + M_{2}) \\
\label{eq:2}
\mu \frac{\partial}{\partial \mu} M_{2} &= \frac{\lambda}{16\pi^{2}} (-4\mu^{2} + 3M_{2} + M_{1}) \\
\mu \frac{\partial}{\partial \mu} m & = \frac{\lambda} {4\pi^{2}} m
\end{align}
It could immediately be obtained that:
\begin{align}
m = m^{0} (\frac{\mu}{\Lambda})^{\lambda / 4\pi^{2}}
\end{align}
In addition, for eq.(\ref{eq:1}) and (\ref{eq:2}), by adding and subtracting both side:
\begin{align}[left=\empheqlbrace]
\label{eq:3}
\mu \frac{\partial}{\partial \mu} (M_{1} + M_{2}) &= \frac{\lambda}{4\pi ^{2}}(M_{1} + M_{2} -2\mu^{2}) \\
\label{eq:4}
\mu \frac{\partial}{\partial \mu} (M_{1} - M_{2}) & = \frac{\lambda}{8\pi^{2}} (M_{1} - M_{2}) 
\end{align}
eq.(\ref{eq:3}) reads, with $a= \frac{\lambda}{16\pi^{2}}$ and $M_{+} = M_{1} + M_{2}$:
\begin{align}
&\frac{\partial M_{+}}{\partial \mu} -4a \frac{M_{+}}{\mu}= -8a\mu   \nonumber \\
\Rightarrow & \mu ^{-4a} \frac{\partial M_{+}}{\partial \mu} - (4a\mu^{-4a-1})M_{+} = -8a\mu^{-4a+1} \nonumber \\
\therefore & \frac{\partial}{\partial \mu} (\mu^{-4a} M) = -8a\mu ^{-4a+1}
\end{align}
Integrating both side:
\begin{align}
\label{eq:5}
M_{+} = M_{+}^{0} (\frac{\mu}{\Lambda})^{4a} + \frac{4a}{2a-1} (\mu^{2} - \Lambda ^{2} (\frac{\mu}{\Lambda})^{4a})
\end{align}
Furthermore, from eq.(\ref{eq:4}):
\begin{align}
\label{eq:6}
M_{-} = M_{-}^0 (\frac{\mu}{\Lambda})^{2a}
\end{align}
Thus, each mass can be written in:
\begin{align}
M_{1} &= \frac{1}{2} [(\frac{\mu}{\Lambda})^{2a} (1+ (\frac{\mu}{\Lambda})^{2a})M_{1} ^{0} + (\frac{\mu}{\Lambda})^{2a} (-1 + (\frac{\mu}{\Lambda})^{2a}) M_{2}^{0} + \frac{4a}{2a-1} (\mu^{2} - \Lambda  ^{2} (\frac{\mu}{\Lambda})^{4a})] \\
M_{2} &= \frac{1}{2} [(\frac{\mu}{\Lambda})^{2a} (-1+ (\frac{\mu}{\Lambda})^{2a})M_{1} ^{0} + (\frac{\mu}{\Lambda})^{2a} (1 + (\frac{\mu}{\Lambda})^{2a}) M_{2}^{0} + \frac{4a}{2a-1} (\mu^{2} - \Lambda  ^{2} (\frac{\mu}{\Lambda})^{4a})] 
\end{align}
\indent Then let us compute the distance between two points $(\varphi_{1} ^{0}, \varphi_{2}^{0})$ and $(\varphi_{1},\varphi _{2})$. First of all, the distance s for unperturbed metric can be computed as follow:
\begin{align}
s &= \int _{0} ^{s} ds = \int \sqrt{G_{ij} \varphi_{i}\varphi_{j}} \nonumber \\
& = \int \sqrt{\frac{k(d\varphi_{1}^{2} + d\varphi_{2}^{2} )}{\varphi_{2}^{2}}} \nonumber \\
&= \int \sqrt{k} \frac{1}{\varphi_{2}} d\varphi _{2} \nonumber \\
\label{eq:7}
&= \sqrt{k} |\ln (\frac{\varphi_{2}} {\varphi_{2} ^{0}})| 
\end{align}
Through the same procedure, the distance for perturbed metric also can be obtained:
\begin{align}
\tilde s &= \int \frac{1}{\varphi_{2}} \sqrt{(k+ \frac{\epsilon}{8\pi^{2}}(\Lambda ^{2} -M_{2}))d\varphi_{1}^{2} + \frac{\epsilon}{4\pi^{2}}d\varphi_{1}d\varphi_{2} + (k + \frac{\epsilon}{8\pi^{2}}(\Lambda ^{2} - M_{1}) )d\varphi_{2} ^{2}} \nonumber \\
&= \int \frac{1}{\varphi_{2}} \sqrt{(k+\frac{\epsilon}{8\pi^{2}})(\Lambda^{2} -M_{2}) (\frac{\epsilon}{4\pi^{2}k}m)^{2} \mp \frac{\epsilon}{4\pi^{2}} \frac{\epsilon}{4\pi^{2}k}m + (k + \frac{\epsilon}{8\pi^{2}}(\Lambda ^{2} - M_{1}))}d\varphi_{2} \nonumber \\
&\approx \sqrt{k}[1 + \frac{1}{2} \lbrace \frac{\epsilon}{8\pi^{2}k} (\Lambda^{2} -M_{1}) + \text{O}(\epsilon^{2})\rbrace] \lvert \ln(\frac{\varphi_{2}}{\varphi_{2}^{0}}) \rvert \nonumber \\
\therefore  \tilde s &= \sqrt{k} [1 + \frac{\epsilon}{16\pi^{2} k} (\Lambda ^{2} -M_{1})] \lvert \ln (\frac{\varphi_{2}} {\varphi_{2} ^{0}}) \rvert  
\end{align}
notice that from first to second line, (\ref{eq:8}) was used.

















\end{document}
