\chapter{Introduction}
\label{Introduction}

One of the biggest goals particle physics has been trying to achieve is to find the most elementary theory of this world. Recently finding the Higgs particle, researchers have constructed successfully the Standard model (SM). However, this is believed as not be a 'theory of everything', but a good approximation which is valid under a certain energy scale. In fact, there are a few phenomenological flaws in the SM such as dark matters. In addition, the SM is the quantum theory of interactions except gravity, and therefore it is natural to assume that it would collapse at Planck scale $M_{p}$, where the effects of gravity is needed to take into account. Practically, it had been believed that such phenomena at extremely high energy can be ignored when we were dealing with characteristic energy scales much smaller than $M_{p}$ according to the idea of effective field theories (EFTs). \\
\indent Nevertheless, recent researches on the string theory and discussions about black holes suggest that requirements of consistency of the physics at Planck scale $M_{p}$, where quatum gravity is needed, likely impose constraints on its low-energy EFT. In other words, it is expected that not every theory which is consistent from the perspective of quantum field theory will be also consistent when quantum gravity is considered. The swampland is the set of the low-energy theories which cannot be compatible with quantum gravity, while subsets of entire low-energy theories coupled to gravity are called the landscape if they are compatible. Those constraints are called the swampland constraints, and lead to possibilities that we should not consider the phenomenological problems at low-energy scales and quantum gravity at ultraviolet scale separately, but think of them as they are related each other. Attempts to discover such constraints, prove (or disprove) and improve them are called the Swampland program of quantum gravity. \\
\indent Since we haven't discovered apparent suggenstions for novel physics while SM has been established experimentally, it has been considered to be useful that we reconsider the belief that SM is the low-energy EFT coupled to gravity, and examine chances if the Swampland program may offer certain hints for particle phenomenology. For this reason, some statements have been proposed to clarify the border separating the swampland and landscape in the space of consistent low-energy EFTs. They are called the Swampland conjectures, because they are still being studied their validities, while some of them are widely accepted as most likely correct. Those conjectures give certain properties qualitatively or quantitatively which the EFTs must obey or avoid in order to realize a consistent completion of low-energy EFTs to quantum gravity. For instance, almost all literatures about the Swampland conjectures start from a claim that there cannot be global symmetries in quantum gravity, that is, any symmetries must be either broken or gauged at high-energy scale. Other examples are such as Weak Gravity Conjecture, Swampland Distance Conjecture, and de Sitter Conjecture. Those conjectures are often supported and motivated  by string theory arguments, while the concepts of the swampland is originally not restricted to string theory. Interestingly, it has seemed from recent studies that some of those conjecures are closely related. Those connections between different statements suggest that they are perhaps alternative aspects of certain unknown properties of quantum gravity. \\
\indent Our interest in this thesis will be on the Swampland Distance Conjecture (SDC). SDC gives a statement about the moduli space, which is controlled by the vacuum expectation value of scalar fields. It claims that in such a space there will be an infinite tower of states which becomes exponentially light at any infinite field distance limit. This is equivalent to say that the cutoff scale for the EFTs decays exponentially. Due to the emergence of an infinite tower of light states, an EFT will collapse, and need to be modified. What we will study about SDC is whether it is valid under the renormalization group (RG). By the computation of RG, the metric of the moduli space will vary, and consequently we have different space. The question we will try to achieve is: Does SDC still hold as a swampland constraint for a new moduli space? Assuming the answer to this question be yes, we will try to find new constraints that theories must satisfies in addition to the original SDC. \\
\indent The structure of the thesis is as follows: In chapter 2, we will introduce the notion of the Swampland programs. Starting from a preliminary of EFTs, more detailed discussions about the swampland and SDC will be provided. Moreover, a brief note on the moduli space, and a focused review of string theory will be presented for the realization of SDC. Chapter 3 will be devoted to another preliminary; renormalization group and EFT. In this chapter, we will review the basic notions of RG, especially Wilsonian renormalization, as well as the RG method we will adopt for the following discussions. In chapter 4, we will compute RG with some simple but non-trivial examples, and study constraints. Finally, we will conclude the discussions. 