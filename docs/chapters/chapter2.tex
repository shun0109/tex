\chapter{Computations}
\label{computations}
In previous chapters, we have all tools for the computaions. In this chapter, we will first provide a simple but non-trivial model, non-linear sigma model. We will also see SDC is satisfied onsuch a model. Then adding various potential terms, the RG computations will be carried out, and by finding the variations of the geodesic distances and assuming SDC will be still valid at varied scale, we may observe new kinds of constraints. 


\section{Non-Linear Sigma Model}
Consider an effective theory with two real scalars $\varphi_{1}$ and $\varphi_{2}$ with its kinetic term given:
\begin{align}
    \frac{k}{\varphi_{2}^{2}} 
\end{align}
where $\sqrt{k}$ is a free parameter. That is, we have an action defiend as:
\begin{align}
    S[\varphi] = \frac{1}{2}\int d^{d}x \sqrt{g} h_{ab}(\varphi) \partial _{\mu} \varphi^{a} \partial ^{\mu} \varphi^{b}
\end{align}
with metric:
\begin{align}
    \dd s^{2} = \frac{k}{\varphi_{2}^{2}} (\dd \varphi_{1}^{2} \dd \varphi_{2}^{2} )
\end{align}
Notice that $k$ determines the Ricci scalar curvature:
\begin{align}
    R = -\frac{2}{k}
\end{align}
Now we consider that SDC is satisfied on this moduli space. In other words, we assume there exists a tower of states with mass scale:
\begin{align}
    M \sim \varphi_{2}^{-a} ,& \qq{} a>0
\end{align}
Besides, curvatures can be computed with formulae:
\begin{align}
    \Gamma _{\mu \nu}^{\lambda} = \frac{1}{2} h^{\lambda \alpha} (h_{\alpha \nu , \mu} + h_{\mu \alpha, \nu} - h_{\mu \nu ,\alpha}) \\
    R_{\alpha \beta \gamma} ^{\sigma} = \Gamma _{\alpha \gamma, \beta} ^{\sigma} - \Gamma _{\beta \gamma , \alpha} ^{\sigma} + \Gamma _{\alpha \gamma} ^{\tau} \Gamma _{\tau \beta} ^{\sigma} - \Gamma_{\alpha \beta}^{\tau} \Gamma_{\tau \gamma} ^{\sigma}
\end{align}
here the summation convention is applied. Non-zero terms are:
\begin{align}
    \label{eq:2.53}
    \Gamma _{12} ^{1} = \Gamma _{21}^{1} = -\frac{1}{\varphi _{2}} \\
    \label{eq:2.54}
    \Gamma _{11} ^{2} = \frac{1}{\varphi _{2}}\\
    \label{eq:2.55}
    \Gamma _{22}^{2} = -\frac{1}{\varphi _{2}}
\end{align}
from which
\begin{align}
    R_{212}^{1} = R_{121}^{2} = -R_{221}^{1} = -R_{112}^{2} = -\frac{1}{\varphi_{2} ^{2}}
\end{align}
These read:
\begin{align}
    R_{1212} = R_{2121} = -\frac{k}{\varphi_{2} ^{4}} ,& \qq{} R_{1221} = R_{2112} = +\frac{k}{\varphi_{2} ^{4}}
\end{align}
Furthermore, the non-vanishing components of Ricci tensor are:
\begin{align}
    R_{11} = R_{22} = -\frac{1}{\varphi_{2}^{2}}
\end{align}
Also, its geodesic equations are:
\begin{align}[left=\empheqlbrace]
    \begin{split}
        \ddot \varphi_{1} - \frac{2}{\varphi_{2}} \dot \varphi_{1} \dot \varphi_{2} &= 0 \\
        \ddot \varphi_{2} +\frac{1}{\varphi_{2}} (\varphi_{1} ^{2} - \varphi_{2}^{2}) &=0
    \end{split}
\end{align}
Thus it can be easily seen that on the $(\varphi_{1},\varphi_{2})$ space, geodesics are either vertical lines or upper-half circles with centers on the $\varphi_{2} = 0$ line. Therefore, the only geodesics approaching the infinite distance limits are vertical lines with $\varphi_{1} = $ constant. Such geodesics are $(\varphi_{1}, \varphi_{2}) =  (\varphi_{1}^{0}, \varphi_{2}^{0} e^{\frac{s}{\sqrt{k}}})$, and thus the distance is:
\begin{align}
    \begin{split}
    s &= \int \sqrt{h_{11} \dd \varphi_{1} ^{2} + h_{22} \dd \varphi_{2} ^{2}} \\
    &= \int \sqrt{k}\frac{\dd \varphi_{2}}{\varphi_{2}} \\
    &= \sqrt{k} \ln \abs{\frac{\varphi_{2}}{\varphi_{2}^{2}}}
    \end{split}
\end{align}
The mass scale of the tower is now:
\begin{align}
    M \sim \exp (-\frac{a}{\sqrt{k}}s) \sim \exp (-\omega s)
\end{align}
then leading to SDC with decay rate $\omega = \frac{a}{\sqrt{k}}$. Indeed, this is $\order {1}$. \\
\indent As before, consider writing the field $\varphi (x)$ in terms of a bckground $\bar {\varphi} (x)$ and a corresponding quantum fluctuation $\xi$ (x). Assume that there is a smooth map $\varphi_{s} (x)$ such that $\varphi _{0} (x) = \bar{\varphi}(x)$, $\varphi_{1} = \varphi (x)$ with $\dot{\varphi} _{0} = \xi$ . Consider a curve $\varphi _{s}$ in the target space which represents the geodesic between initial and final points, i.e. $\varphi _{0}$ and $\varphi _{1}$:
\begin{align}
    \ddot{\varphi}_{s} ^{a} (x) + \Gamma _{bc}^{a} \dot{\varphi} _{s}^{b}(x) \dot{\varphi} _{s}^{c} (x)  
\end{align}
where $\Gamma$ denotes the Christoffel symbol corresponding to the target space metric $h_{ab}$. Then the expansion of the action around the background $\bar{\varphi}$ is given:
\begin{align}
    S[\varphi] &=\eval {\sum_{n=0}^{\infty} \frac{1}{n!} \dv[n] {s} S[\varphi _{s}]} _{s=0} \nonumber \\
    & = \eval {\sum_{n=0}^{\infty} (\nabla _{s})^{n} S[\varphi _{s}]} _{s=0}
\end{align}
where $\nabla _{s}$ is the covariant derivative along the curve $\varphi _{s} ^{a}$. Using identities:
\begin{align}
    &\nabla_{s} \partial _{\mu} \varphi ^{i} = \partial _{\mu} \dv{\varphi^{i}}{s} + \partial_{\mu} \varphi ^{k} \Gamma _{kj}^{i} \dv{\varphi^{j}}{s} = \nabla _{\mu} \xi ^{i} 
    & \nabla _{s} h_{ab} = 0 \\
    & \nabla _{s} \dv{\varphi ^{i}} {s} = 0 \\
    &[\nabla _{s}, \nabla _{\mu}] Z^{k} = \dv{\varphi ^{i}}{s} \partial _{\mu} \varphi ^{j} R_{lij} ^{k} Z^{l}
\end{align}
with $Z^{i}$ is an arbitrary vector, it is followed by:
\begin{align}
    \begin{split}
    S[\varphi] = & \frac{1}{2} \int d^{d}x \sqrt{g} h_{ab}(\bar{\varphi}) \partial _{\mu} \bar{\varphi}^{a} \partial ^{\mu} \bar{\varphi} ^{b} + \int d^{d}x \sqrt{g} h_{ab} \partial _{\mu} \xi ^{a} \partial ^{\mu} \xi ^{b} \\
    &+ \frac{1}{2} \int d^{d}x \sqrt{g} \lbrace h_{ab} \nabla_{\mu} \xi^{a} \nabla ^{\mu} \xi ^{b} + R_{acdb} \xi ^{c}\xi^{d} \partial _{\mu}\bar{\varphi}^{a} \partial ^{\mu} \bar{\varphi} ^{b} + \cdots
    \end{split} 
\end{align}
Looking at the quadratic part, it lead to an operator:
\begin{align}
    D = -(h_{ab} \square + R_{acbd} \partial _{\mu} \bar{\varphi}^{c} \partial ^{\mu} \bar{\varphi}^{d} )
\end{align}
Therefore, for instance in $d=4$ case, the 1-loop effective action is given:
\begin{align}
    \begin{split}
    \Gamma _{\text{1-loop}} ^{\text{div}} = &-\frac{1}{2} \frac{1}{(4\pi)^2}\lbrace \frac{1}{2}(\Lambda^4 - \mu ^{4}) \int d^{4}x \sqrt{g} + (\Lambda ^{2} - \mu ^{2}) \int d^{4}x \sqrt{g} \Tr R_{acbd} \partial _{\mu} \bar{\varphi}^{c} \partial ^{\mu} \bar{\varphi}^{d} \\ 
    &+ \frac{1}{2} \ln (\Lambda ^{2} / \mu ^{2})\int d^{4}x  \sqrt{g} \Tr (R_{acbd} \partial _{\mu} \bar{\varphi}^{c} \partial ^{\mu} \bar{\varphi}^{d}) ^{2} \rbrace
    \end{split}
\end{align}
Thus, comparing only relevant coefficients:
\begin{align}
    \mu \pdv{\mu} g_{ab} = \frac{\mu ^{2}} {(4\pi)^2} R_{ab}
\end{align}
Adding a generic potential term $V$, the operator reads:
\begin{align}
    D = -(h_{ab}\square + R_{acbd} \partial_{\mu} \bar{\varphi} ^{a} \partial^{\mu} \bar{\varphi} ^{b} - \text{Hess} (V))
\end{align}
where $\text{Hess}(V)$ is a Hessian matrix of V. Then the divergent part of 1-loop effective action in 4d becomes:
\begin{align}
    \begin{split}
    \Gamma_{\text{1-loop}}^{\text{div}} = &-\frac{1}{2} \frac{1}{(4\pi)^{2}}\lbrace (\Lambda ^{2} - \mu ^{2})\int d^{4}x \sqrt{g} \Tr (R_{acbd} \partial_{\mu} \bar{\varphi} ^{a} \partial^{\mu} \bar{\varphi} ^{b} - \text{Hess} (V)) \\ 
    &+ \ln (\Lambda/\mu)\int d^{4}x \sqrt{g} \Tr(R_{acbd} \partial_{\mu} \bar{\varphi} ^{a} \partial^{\mu} \bar{\varphi} ^{b} - \text{Hess} (V))^{2}
    \end{split}
\end{align}
By comparing coefficients with an action at scale $\mu$:
\begin{align}
    S[\varphi] = \int d^{4}x \sqrt{g} (\frac{1}{2} \tilde h_{ab} \partial _{\mu} \varphi^{a} \partial ^{\mu} \varphi ^{b} + V)
\end{align}
flow equations can be obtained as did above.

\subsection{Generic Mass}
Suppose that the potential is given in a form of the homogeneous quadratic function of $\varphi_{1}$ and $\varphi _{2}$:
\begin{align}
    V= \frac{1}{2} \sum_{i,j} M_{ij} \varphi^{i} \varphi^{j}
\end{align}
Hence, its Hessian matrix is:
\begin{align}
    Hess(V) = 
    \begin{bmatrix}
        M_{1} & m \\
        m & M_{2}
    \end{bmatrix}
\end{align}
here $M_{11} = M_1$, $M_{22} = M_{2}$ and $M_{12} = M_{21} = m$. Then The relevant divergence of 1-loop effective action is:
\begin{align}
    \begin{split}
        \Gamma _{\text{1-loop}}^{\text{div}} = & -\frac{1}{2} \frac{1}{(4\pi)^2} \int d^{4}x \sqrt{g} \lbrace (\Lambda ^{2} - \mu ^{2})(R_{cd} \partial _{\mu} \varphi ^{c} \partial ^{\mu} \varphi ^{d} - \Tr Hess(V)) \\
        & + \ln (\Lambda ^{2} / \mu ^{2}) (\frac{1}{2} \Tr Hess(V)^{2}  - \Tr Hess(V) R_{acbd} \partial _{\mu} \varphi ^{c} \partial ^{\mu} \varphi ^{d})
    \end{split}
\end{align}
Therefore, the metric $h_{ab}$ flows as:
\begin{align}
    \mu \pdv{\mu} h_{ab} = \frac{1}{8\pi^2} (\mu ^{2} R_{ab} - \Tr Hess(V) R_{acbd})
\end{align}
Now considering how the metric, and geodesics eventually, would be modified as the scale varies from $\Lambda$ to $\mu$, integrate again both side, it becomes:
\begin{align}
    \label{eq:2.63}
    h_{ab} ^{\mu} - h_{ab} ^{\Lambda} = \frac{1}{8\pi^{2}}\lbrace \frac{1}{2}(\mu ^{2} - \Lambda ^{2}) R_{ab} - \ln \frac{\mu}{\Lambda} \Tr Hess(V) R_{acbd} \rbrace
\end{align}
Assume here that the variation of the scale is so small that the curvature on the right hand side can be thought of as the one corresponding to the initial metric at scale $\Lambda$, writing $\mu = \Lambda (1-\epsilon)$ with $\epsilon \ll 1$, eq.\ref{eq:2.63} gives:
\begin{align}
    h_{ab} ^{\mu} \approx h_{ab} ^{\Lambda} + \frac{\epsilon}{8\pi ^{2}}  \qty (\Tr Hess(V) R_{acbd} - \Lambda ^{2} R_{ab})
\end{align}
For given metric and a form of the potential, each term is:
\begin{align}
    \Tr Hess(V) R_{acbd} &= \frac{1}{k} 
    \begin{bmatrix}
        -M_{2} & m \\
        m & -M_{1}
    \end{bmatrix} \\
    R_{ab} &= -\frac{1}{\varphi_{2} ^{2}} \mathbb{1}
\end{align}
the metric at a scale $\mu$ is:
\begin{align}
    \tilde h_{ab} := h_{ab} ^{\mu} = 
    \begin{bmatrix}
        \frac{k}{\varphi _{2} ^{2}} - \frac{\epsilon}{8\pi^{2}} \lbrace \frac{M_{2}}{k} - (\frac{\Lambda}{\varphi_{2}})^{2} \rbrace & \frac{\epsilon}{8\pi^{2}} \frac{m}{k} \\
        \frac{\epsilon}{8\pi^{2}} \frac{m}{k} & \frac{k}{\varphi _{2} ^{2}} - \frac{\epsilon}{8\pi^{2}} \lbrace \frac{M_{1}}{k} - (\frac{\Lambda}{\varphi_{2}})^{2} \rbrace
    \end{bmatrix}
\end{align}
and its inverse is:
\begin{align}
    \tilde h^{ab} = 
    \begin{bmatrix}
        \frac{\varphi_{2}^{2}}{k} + \frac{\epsilon}{8\pi^{2}}(\frac{\varphi_{2}^{2}}{k})^{2} \lbrace \frac{M_{2}}{k} - (\frac{\Lambda}{\varphi_{2}})^{2} \rbrace & -\frac{\epsilon}{8\pi^{2}} \frac{m}{k}(\frac{\varphi_{2}^{2}}{k})^{2} \\
        -\frac{\epsilon}{8\pi^{2}} \frac{m}{k}(\frac{\varphi_{2}^{2}}{k})^{2} & \frac{\varphi_{2}^{2}}{k} + \frac{\epsilon}{8\pi^{2}}(\frac{\varphi_{2}^{2}}{k})^{2} \lbrace \frac{M_{1}}{k} - (\frac{\Lambda}{\varphi_{2}})^{2} \rbrace
    \end{bmatrix}
\end{align}

Now in order to investigate how this geodesic varies as there is a tiny variation on the metric. From perturbed metric, one may find:
\begin{align}
    &\tilde \Gamma_{11}^{1} = -\frac{\epsilon}{8\pi^{2}} \frac{m}{k^2}\varphi_{2} \\
    &\tilde \Gamma _{12} ^{1} = -\frac{1}{\varphi_{2}} -\frac{\epsilon}{8\pi^{2}}\frac{M_2}{k^2}\varphi_{2} \\
    &\tilde \Gamma _{22}^{1} = \frac{\epsilon}{8\pi^{2}} \frac{m}{k^2} \varphi_{2} \\
    &\tilde \Gamma _{11} ^{2} = \frac{1}{\varphi_{2}} + \frac{\epsilon}{8\pi^{2}}\frac{M_{1}}{k^{2}}\varphi_{2}\\
    &\tilde \Gamma _{12} ^{2} = \frac{\epsilon}{8\pi ^{2}} \frac{m}{k^{2}}\varphi_{2} \\
    &\tilde \Gamma _{22} ^{2} = -\frac{1}{\varphi_{2}} -\frac{\epsilon}{8\pi^{2}}\frac{M_1}{k^{2}}\varphi_{2}
\end{align}
Therefore, the new geodesic equations are:
\begin{align}[left=\empheqlbrace]
    \label{eq:2.79}
    \ddot \varphi_{1} - \frac{2}{\varphi_{2}} \dot \varphi_{1} \dot \varphi_{2} &= \frac{\epsilon}{8\pi ^{2}} \frac{\varphi_{2}}{k^{2}} (m\dot\varphi_{1}^{2} + 2M_{2} \dot \varphi_{1} \dot \varphi_{2} - m\dot \varphi_{2}^{2} )\\
    \label{eq:2.80}
    \ddot \varphi_{2} +\frac{1}{\varphi_{2}} (\varphi_{1} ^{2} - \varphi_{2}^{2}) &= \frac{\epsilon}{8\pi ^{2}}\frac{\varphi_{2}}{k^{2}}(-M_{1} \dot \varphi_{1}^{2} - 2m \dot \varphi_{1} \dot \varphi_{2} + M_{1} \dot \varphi_{2} ^{2})
\end{align}
Notice that $\dot \varphi_{1} = 0$ is no longer a solution of the equations. Then assuming an ansatz such that the solutions for those geodesic equations $(\varphi_{1}, \varphi{2})$ are in the form:
\begin{align}[left=\empheqlbrace]
    \varphi_{1} &= \tilde \varphi_{1} + \epsilon f(s) \\
    \varphi_{2} &= \tilde \varphi_{2} + \epsilon g(s) 
\end{align}
where tildes denote the unperturbed solutions, with constraints:
\begin{align}
    \eval{\varphi_{i} } _{s=0} = \tilde \varphi_{i} ,& \qq{} \eval{\dot \varphi_{i}} _{s=0} = \dot \tilde\varphi_{i} 
\end{align}
Knowing $\tilde \varphi_{1} = \varphi_{1}^{0} =$ constant, and $\tilde \varphi_{2} = \varphi_{2} ^{0} e^{s/\sqrt{k}}$, eq.\ref{eq:2.79} leads:
\begin{align}
    &(\tilde \varphi_{2} + \epsilon g(s))(\ddot {\tilde {\varphi_{1}}} + \epsilon \ddot f(s)) - 2 (\dot {\tilde{\varphi_{1}}} + \epsilon \dot {f}(s))(\dot {\tilde{\varphi_{2}}} +\epsilon \dot{g}(s)) \nonumber \\
    &= \frac{\epsilon}{8\pi^{2}}\frac{1}{k^2} (\tilde{\varphi_{2}} + \epsilon g(s))^{2}\lbrace m(\dot{\tilde{\varphi_{2}}} + \epsilon \dot{f}(s))^{2} +2M_{2} (\dot{\tilde{\varphi_{1}}} + \epsilon \dot{f}(s))(\dot{\tilde{\varphi_{2}}} + \epsilon \dot{g}(s)) - m (\dot{\tilde{\varphi_{2}}} + \epsilon \dot{g}(s))^{2} \rbrace \nonumber \\
    \Rightarrow &\tilde{\varphi_{2}}\ddot{\tilde{\varphi_{1}}}-2\dot{\tilde{\varphi_{1}}}\dot{\tilde{\varphi_{2}}} + \epsilon\lbrace \tilde{\varphi_{2}}\ddot{f}(s) + \ddot{\tilde{\varphi_{1}}}g(s) - 2(\dot{\tilde{\varphi_{1}}}\dot{g}(s) + \dot{\tilde{\varphi_{2}}}\dot{f}(s)) \rbrace  + \mathcal{O}(\epsilon ^{2}) \nonumber \\
    &= \frac{\epsilon}{8\pi^{2}}\frac{1}{k^{2}} \tilde{\varphi _{2}} ^{2} (m\dot{\tilde{\varphi_{2}}} ^{2} + 2M_{2} \dot{\tilde{\varphi_{1}}}\dot{\tilde{\varphi_{2}}} -m \dot{\tilde{\varphi_{2}}}^{2}) + \mathcal{O}(\epsilon^{2}) \nonumber
\end{align}
Then it ends up with a differential equation:
\begin{align}
    \label{eq:2.84}
    \ddot{f} (s) - \frac{2}{\sqrt{k}} \dot {f} (s)+ \frac{1}{8\pi^{2}} \frac{m}{k^{3}}\tilde \varphi_{2} ^{3} = 0 
\end{align}
A solution to the homogeneous equation is:
\begin{align}
    f_{h} (s) = C_{1} + C_{2} e^{2s/\sqrt{k}} 
\end{align}
and let $f_{p} = A e^{3s/ \sqrt{k}}$ be the particular solution, then since $\ddot f_{p} (s) = \frac{9}{k} Ae^{3s/\sqrt{k}}$, $\dot f_{p} (s) = \frac{3}{\sqrt{k}}$, $A = -\frac{k}{3} \frac{1}{8\pi^{2}}\frac{m}{k^{3}} (\varphi_{2}^{0})^{3}$. Therefore:
\begin{align}
    f(s) &= f_{h}(s) + f_{p}(s) \nonumber \\
    &= C_{1} + C_{2} e^{2s/\sqrt{k}} - \frac{k}{3}\frac{1}{8\pi^{2}}\frac{m}{k^{3}}(\varphi_{2}^{0})^{3} e^{3s/\sqrt{k}}
\end{align}
Considering its initial conditions:
\begin{align}
    f(0) &= C_{1} + C_{2} - \frac{1}{3}\frac{m}{8\pi^{2}k^{2}}(\varphi_{2}^{0})^{3} \\
    \dot{f}(0) &= \frac{2}{\sqrt{k}}C_{2} -  \frac{1}{8\pi^{2}}\frac{m}{k^{2}\sqrt{k}}(\varphi_{2}^{0})^{3}
\end{align}
the correction for $\varphi_{1}$ is given:
\begin{align}
    f(s) = -\frac{1}{8\pi^{2}}\frac{m}{6k^{2}}(\varphi_{2}^{0})^{3}(1-3e^{\frac{2s}{\sqrt{k}}} + 2e^{\frac{3s}{\sqrt{k}}})
\end{align}
Also, for eq.\ref{eq:2.80}:
\begin{align}
    &\tilde{\varphi_{2}}\ddot{\tilde{\varphi_{2}}} -\dot{\tilde{\varphi_{2}}}^{2} + \epsilon(\varphi_{2}\ddot{g}(s) + \ddot{\tilde{\varphi_{2}}} g(s) - 2\dot{\tilde{\varphi_{2}}}\dot{g}(s)) = \frac{\epsilon}{8\pi^{2}}\frac{M_1}{k^2} \tilde{\varphi_{2}}^2 \dot{\tilde{\varphi_{2}}}^2 \\
    \Rightarrow & \ddot{g}(s) - \frac{2}{\sqrt{k}} \dot{g}(s) +\frac{1}{k}g(s) - \frac{1}{8\pi^{2}}\frac{M_{1}}{k^3} (\varphi_{2}^{0})^{3} e^{\frac{3s}{\sqrt{k}} }= 0
\end{align}
Through the same procedure to $f(s)$, the correction for $\varphi_{2}$ is:
\begin{align}
    g(s) = \frac{1}{8\pi^{2}} \frac{M_{1}}{4k^{5/2}}(\varphi_{2}^{0})^{3}e^{s/\sqrt{k}} (-\sqrt{k} - 2s +\sqrt{k} e^{2s/\sqrt{k}})
\end{align}
Therefore, up to $\order {\epsilon}$, corrected geodesics are:
\begin{align}[left=\empheqlbrace]
    \varphi_{1} &= \varphi_{1}^{0} - \frac{\epsilon}{8\pi^2} \frac{m}{6k^{2}} (\varphi_{2}^{0})^{3} (1-3e^{\frac{2s}{\sqrt{k}}} + 2e^{\frac{3s}{\sqrt{k}}}) \\
     \varphi_{2} &= \varphi_{2}^{0} e^{\frac{s}{\sqrt{k}}} + \frac{\epsilon}{8\pi^{2}}\frac{M_{1}}{4k^{5/2}} (\varphi_{2}^{0})^{3} e^{\frac{s}{\sqrt{k}}} (-\sqrt{k} - 2s + \sqrt{k} e^{\frac{2s}{\sqrt{k}}})
\end{align}
Finally, from those information, variations of a distance between two points as the scale variation can be also investigated. A distance for corrected geodesics is:
\begin{align}
    s &= \int \sqrt{\tilde h_{ij} \dd \varphi_{i} \dd \varphi_{j}} \nonumber \\
    &= \int \sqrt{[\frac{k}{\varphi_{2}^{2}} -\frac{\epsilon}{8\pi^{2}}\lbrace \frac{M_2}{k} - (\frac{\Lambda}{\varphi_{2}})^{2}\rbrace ]\dd \varphi_{1}^{2} + \frac{\epsilon}{4\pi^{2}} \frac{m}{k}\dd \varphi_{1} \dd \varphi_{2} +[\frac{k}{\varphi_{2}^{2}} -\frac{\epsilon}{8\pi^{2}}\lbrace \frac{M_1}{k} - (\frac{\Lambda}{\varphi_{2}})^{2}\rbrace ] \dd \varphi_{2}^{2} } \nonumber
\end{align}
However, from corrected geodesics, it can be deduced that:
\begin{align} 
    \begin{split}
    \dv{\varphi_{1}}{\varphi_{2}} &= \dv{\varphi_{1}}{\tilde{\varphi_{2}}}\dv{\tilde{\varphi_{2}}}{\varphi_{2}}  \\
    &= -\frac{\epsilon}{8\pi^{2}}\frac{m}{6k^{2}} (-6\varphi_{2}^{0} \tilde{\varphi_{2}} + 6 \tilde{\varphi_{2}}^{2})(1+\frac{\epsilon}{8\pi^{2}} \frac{M_{1}}{4k^{5/2}} \\
    & (-\sqrt{k}(\varphi_{2}^{0} )^{2} - 2s(\varphi_{2}^{0}) ^{2} + 3\sqrt{k} \tilde{\varphi_{2}}^{2}))^{-1}  \\
    &= -\frac{\epsilon}{8\pi^{2}}\frac{m}{k^{2}} (\tilde{\varphi_{2}} - \varphi_{2}^{0})\tilde{\varphi_{2}} + \mathcal{O}(\epsilon ^{2})
    \end{split}
\end{align}
With this aid, the distance can be in the form:
\begin{align}
    \begin{split}
    s &=\int [[\frac{k}{\varphi_{2}^{2}} -\frac{\epsilon}{8\pi^{2}}\lbrace \frac{M_2}{k} - (\frac{\Lambda}{\varphi_{2}})^{2}\rbrace] \lbrace -\frac{\epsilon}{8\pi^{2}} \frac{m}{k^{2}} \tilde{\varphi_{2}}(\tilde{\varphi_{2}} -\varphi_{2}^{0}) \rbrace ^{2} \\
    & + \frac{\epsilon}{4\pi^{2}}\frac{m}{k}\lbrace -\frac{\epsilon}{8\pi^{2}} \frac{m}{k^{2}} \tilde{\varphi_{2}}(\tilde{\varphi_{2}} -\varphi_{2}^{0}) \rbrace  \\
    & + [\frac{k}{\varphi_{2}^{2}} -\frac{\epsilon}{8\pi^{2}}\lbrace \frac{M_1}{k} - (\frac{\Lambda}{\varphi_{2}})^{2}\rbrace]]^{1/2} \dd {\varphi_{2}}  \\
    &\sim \int \frac{\sqrt{k}}{\varphi_{2}} [1- \frac{\epsilon}{16\pi^{2}} \frac{\varphi_{2}^{2}}{k} \lbrace\frac{M_{1}}{k} - (\frac{\Lambda}{\varphi_{2}})^{2} \rbrace ] + \order*{\epsilon^{2}} \dd{\varphi_{2}} \\
    &= \sqrt{k} \ln\abs*{\frac{\varphi_{2}}{\varphi_{2}^{0}}} - \frac{\epsilon}{16\pi^{2}}\lbrace \frac{M_{1}}{2k\sqrt{k}}(\varphi_{2}^{2} - (\varphi_{2}^{0})^{2}) - \frac{\Lambda^{2}}{\sqrt{k}}\ln\abs*{\frac{\varphi_{2}}{\varphi_{2}^{0}}} \rbrace + \order*{\epsilon ^{2}}
    \end{split}
\end{align}
It can be easily noticed that in the limit $\epsilon \to 0$, $\varphi_{2} \to \tilde \varphi_{2}$ and thus the distance for the original geodesic is restored. Also, notice that SDC is violated since for large enough $\varphi_{2}$:
\begin{align}
    \dv{\ln M}{s} = \dv{\ln M}{\varphi_{2}}\dv{\varphi_{2}}{s} = -\frac{\lambda}{\sqrt{k}}\abs{\frac{\epsilon}{8\pi^{2}} \frac{M_{1}}{k^{2}}\varphi_{2}^{2}}^{-1/2} \rightarrow 0
\end{align}
so the tower mass scale is no longer decaying exponentially with the geodesic distance. This is compatible with SDC if and only if the coefficient of the polynomial term vanishes. In this case, we have a new constraint $M_{1} = 0$. 

\subsection{Quartic Coupling}
Next, let us add a quartic coupling term to the action. Then the potential has a form:
\begin{align}
    V= \frac{1}{2}M_{ij} \varphi_{i} \varphi_{j} + \frac{\lambda}{4} (\varphi_{1}^{2} + \varphi_{2}^{2})^{2}
\end{align}
Then its Hessian matrix is:
\begin{align}
    Hess(V) = \begin{bmatrix}
        M_{1} & m \\
        m & M_{2}
    \end{bmatrix}
    + \lambda
    \begin{bmatrix}
        3\varphi_{1}^{2} + \varphi_{2} ^{2} & 2\varphi_{1}\varphi_{2} \\
        2\varphi_{1}\varphi_{2} & \varphi_{1}^{2} + 3\varphi_{2} ^{2}
    \end{bmatrix}
\end{align}
Hence, the metric becomes:
\begin{align}
    \tilde{h}_{ab} = 
    \begin{bmatrix}
        \frac{k}{\varphi_{2}^{2}} - \frac{\epsilon}{8\pi^{2}}[\frac{1}{k} \lbrace M_{2} +\lambda (\varphi_{1}^{2} + 3\varphi_{2}^{2})\rbrace - (\frac{\Lambda}{\varphi_{2}})^{2}] & \frac{\epsilon}{8\pi^{2}}\frac{1}{k} (m+2\lambda\varphi_{1}\varphi_{2}) \\
        \frac{\epsilon}{8\pi^{2}}\frac{1}{k} (m+2\lambda\varphi_{1}\varphi_{2}) & \frac{k}{\varphi_{2}^{2}} - \frac{\epsilon}{8\pi^{2}}[\frac{1}{k} \lbrace M_{1} +\lambda (3\varphi_{1}^{2} + \varphi_{2}^{2})\rbrace - (\frac{\Lambda}{\varphi_{2}})^{2}]
    \end{bmatrix}
\end{align}
Then we may have:
\begin{align}
    \Gamma_{11}^{1} &= -\frac{\epsilon}{8\pi^{2}}\frac{\varphi_{2}}{k^{2}}(m+3\lambda\varphi_{1}\varphi_{2}) \\
    \Gamma_{12}^{1} &= -\frac{1}{\varphi_{2}} - \frac{\epsilon}{8\pi^{2}}\frac{\varphi_{2}}{k^{2}}(M_{2} + \lambda(\varphi_{1}^{2} + 6\varphi_{2}^{2})) \\
    \Gamma_{22}^{1} &= \frac{\epsilon}{8\pi^{2}} \frac{\varphi_{2}}{k^{2}}(m+6\lambda\varphi_{1}\varphi_{2})\\
    \Gamma_{11}^{2} &= \frac{1}{\varphi_{2}} + \frac{\epsilon}{8\pi^{2}}\frac{\varphi_{2}}{k^{2}}[M_{1} + \lambda (3\varphi_{1}^{2} + 6\varphi_{2}^{2})] \\
    \Gamma_{12}^{2} &= \frac{\epsilon}{8\pi^{2}} \frac{\varphi_{2}}{k^{2}}(m-\lambda\varphi_{1}\varphi_{2}) \\
    \Gamma_{22}^{2} &= -\frac{1}{\varphi_{2}} - \frac{\epsilon}{8\pi^{2}}\frac{\varphi_{2}}{k^{2}}[M_{1} + \lambda (3\varphi_{1}^{2} + 2\varphi_{2}^{2})]
\end{align}
With same procedure as before, we can obtain the geodesic equations:
\begin{align}
    \label{eq:104}
    \begin{split}
    \ddot \varphi_{1} - \frac{2}{\varphi_{2}} \dot \varphi_{1} \dot \varphi_{2} &= \frac{\epsilon}{8\pi^{2}} \frac{\varphi_{2}}{k^{2}}[(m+3\lambda\varphi_{1}\varphi_{2})\dot \varphi_{1}^{2} + 2\lbrace M_{2} + \lambda(\varphi_{1}^{2} + 6\varphi_{2}^{2})\rbrace \dot \varphi_{1}\varphi_{2}  \\
    & - (m+6\lambda\varphi_{1}\varphi_{2})\dot \varphi_{2} ^{2}]
    \end{split}
\end{align}
\begin{align}
    \label{eq:105}
    \begin{split}
    \ddot \varphi_{2} + \frac{1}{\varphi_{2}} (\dot \varphi_{1}^{2} - \dot \varphi_{2}^{2}) &= \frac{\epsilon}{8\pi^{2}}\frac{\varphi_{2}}{k^{2}} [-\lbrace M_{1} +\lambda(3\varphi_{1}^{2} + 6\varphi_{2}^{2})\rbrace \dot \varphi_{1}^{2} - 2(m-\lambda\varphi_{1} \varphi_{2})\dot \varphi_{1} \dot \varphi_{2}  \\
    & + \lbrace M_{1} + \lambda (3\varphi_{1}^{2} + 2\varphi_{2}^{2}) \rbrace \dot \varphi_{2}^{2}]
    \end{split}
\end{align}
Then, eq.\ref{eq:104} gives:
\begin{align}
    \ddot f (s) - \frac{2}{\sqrt{k}} \dot f(s) = -\frac{1}{8\pi^{2}}\frac{1}{k^{3}}(m+6\lambda \tilde{\varphi_{1}} \tilde{\varphi_{2}})\tilde{\varphi_{2}}^{3}
\end{align}
where $f(s)$, and $\tilde{\varphi_{i}}$ are same as we did. Solving for $f(s)$, we end up with:
\begin{align}
    \varphi_{1} = \varphi_{1}^{0} - \frac{\epsilon}{8\pi^{2}} \frac{m}{6k^{2}} (\varphi_{2}^{0})^{3} (1-3e^{\frac{2s}{\sqrt{k}}} + 2e^{\frac{3s}{\sqrt{k}}}) - \frac{\epsilon}{8\pi^{2}}\frac{3\lambda}{4k^{2}}\varphi_{1}^{0} (\varphi_{2}^{0})^{4} (1-2e^{\frac{2s}{\sqrt{k}}} + e^{\frac{4s}{\sqrt{k}}})
\end{align}
Similarly, eq.\ref{eq:105} reads:
\begin{align}
    \ddot g(s) - \frac{2}{\sqrt{k}} \dot g(s) +\frac{1}{k}g(s) = \frac{1}{8\pi^{2}}\frac{1}{k^{3}} (M_{1} + 3\lambda \tilde{\varphi_{1}}^{2} + 2\lambda \tilde{\varphi_{2}}^{2})\tilde{\varphi_{2}} ^{3}
\end{align}
and we eventually find:
\begin{align}
    \begin{split}
        \varphi_{2}  = & \varphi_{2} ^{0} e^{\frac{s}{\sqrt{k}}} + \frac{\epsilon}{8\pi^{2}}\frac{1}{4k^{5/2}}(\varphi_{2}^{0})^{3} e^{\frac{s}{\sqrt{k}}} (-\sqrt{k} - 2s + \sqrt{k} e^{\frac{2s}{\sqrt{k}}})(M_{1} + 3\lambda (\varphi_{1}^{0})) \\
        & + \frac{\epsilon}{8\pi^{2}} \frac{\lambda}{4k^{5/2}} (\varphi_{2}^{0})^{5} (-\frac{\sqrt{k}}{2} - 2s + 2\sqrt{k}e^{\frac{4s}{\sqrt{k}}})
    \end{split}
\end{align}
In order to determine the geodesic distance, we again manipulate first:
\begin{align}
    \dv{\varphi_{1}}{\varphi_{2}} = \dv{\varphi_{1}}{\tilde{\varphi_{2}}}\dv{\tilde{\varphi_{2}}}{\varphi_{2}} = -\frac{\epsilon}{8\pi^{2}}[\frac{m}{k^{2}}(\tilde{\varphi_{2}} - \varphi_{2}^{0})\tilde{\varphi_{2}} + \frac{3\lambda}{k^{2}}\varphi_{1}^{0} (\tilde{\varphi_{2}}^{2} - (\varphi_{2}^{0})^{2})\tilde{\varphi_{2}}] + \order{\epsilon ^{2}} 
\end{align}
Thus, the distance is given:
\begin{align}
    \begin{split}
    s &= \int \sqrt{\tilde{h}_{ab} \dd \varphi_{a} \dd \varphi_{b}} \\
    & \sim \sqrt{k}\ln \abs{\frac{\varphi_{2}}{\varphi_{2}^{0}}} - \frac{\epsilon}{16\pi^{2}}\frac{1}{\sqrt{k}} [\frac{1}{2k}(M_{1}+3\lambda(\varphi_{1}^{0})^{2})(\varphi_{2}^{2} - (\varphi_{2}^{0})^{2}) + \frac{\lambda}{4k} (\varphi_{2}^{4} - (\varphi_{2}^{0})^{4}) + \Lambda^{2} \ln \abs{\frac{\varphi_{2}}{\varphi_{2}^{0}}}]
    \end{split}
\end{align}
Notice that we have two polynomial terms here. Assuming SDC is always satisfied at different energy scales, we arrive the constraints:
\begin{align}
    \lambda = 0, & \qq{} M_{1} = 0
\end{align}

\subsection{Exponential Mass}
Now suppose $M_{1} = A e^{-\alpha \varphi_{1}}$. Then the Hessian is:
\begin{align}
    Hess(V) = 
    \begin{bmatrix}
        Ae^{-\alpha \varphi_{1}}(\alpha^{2}\varphi_{1}^{2} - 4\alpha\varphi_{1} + 2) & m \\
        m & M_2
    \end{bmatrix}
\end{align}
and the perturved metric is:
\begin{align}
    \tilde h_{ab} = 
    \begin{bmatrix}
        \frac{k}{\varphi_{2}^{2}} -\frac{\epsilon}{8\pi^{2}}[\frac{M_{2}}{k} - (\frac{\Lambda}{\varphi_{2}})^{2}] & \frac{\epsilon}{8\pi^{2}}\frac{m}{k} \\
        \frac{\epsilon}{8\pi^{2}}\frac{m}{k} & \frac{k}{\varphi_{2}^{2}} -\frac{\epsilon}{8\pi^{2}} [\frac{Ae^{-\alpha\varphi_{1}}}{k}(\alpha^{2} \varphi_{1}^{2} - 4\alpha\varphi_{1} + 2) - (\frac{\Lambda}{\varphi_{1}})^{2}]
    \end{bmatrix}
\end{align}
Then:
\begin{align}
    \Gamma_{11}^{1} &= -\frac{\epsilon}{8\pi^{2}}\frac{\varphi_{2}}{k^{2}}m \\
    \Gamma_{12}^{1} &= -\frac{1}{\varphi_{2}} - \frac{\epsilon}{8\pi^{2}}\frac{\varphi_{2}}{k^{2}}M_{2} \\
    \Gamma_{22}^{1} &= \frac{\epsilon}{8\pi^{2}} \frac{\varphi_{2}}{k^{2}}[\frac{A}{2}\varphi_{2} e^{-\alpha\varphi_{1}} (-\alpha^{3}\tilde{\varphi_{1}}^{2} + 6\alpha^{2} \tilde{\varphi_{1}} - 6\alpha)+m]\\
    \Gamma_{11}^{2} &= \frac{1}{\varphi_{2}} + \frac{\epsilon}{8\pi^{2}}\frac{A\varphi_{2}}{k^{2}}e^{-\alpha\varphi_{1}}(\alpha^{2}\varphi_{1}^{2} - 4\alpha\varphi_{1} +2) \\
    \Gamma_{12}^{2} &= \frac{\epsilon}{8\pi^{2}} \frac{\varphi_{2}}{k^{2}}(m - \frac{A}{2} \varphi_{2} e^{-\alpha \varphi_{1}} (-\alpha^{3} \varphi_{1}^{2} + 6\alpha^{2} \varphi_{1} -6\alpha)) \\
    \Gamma_{22}^{2} &= -\frac{1}{\varphi_{2}} - \frac{\epsilon}{8\pi^{2}}\frac{A\varphi_{2}}{k^{2}}e^{-\alpha\varphi_{1}} (\alpha^{2} \varphi_{1}^{2} -4\alpha \varphi_{1} +2)
\end{align}
and geodesic equations are:
\begin{align}
    \begin{split}
    \ddot \varphi_{1} - \frac{2}{\varphi_{2}} \dot \varphi_{1} \dot \varphi_{2} & = \frac{\epsilon}{8\pi^{2}}\frac{\varphi_{2}}{k^{2}} [m\dot \varphi_{1}^{2} + 2M_{2} \dot \varphi_{1} \dot \varphi_{2} - \frac{A}{2} \varphi_{2} e^{-\alpha\varphi_{1}} (-\alpha^{3} \varphi_{1} ^{2} + 6\alpha^{2} \varphi_{1} -6\alpha)\dot \varphi_{2}^{2}  \\
    & - m\dot \varphi_{2}^{2}]
    \end{split}
\end{align}
\begin{align}
    \begin{split}
    \ddot \varphi_{2} + \frac{1}{\varphi_{2}} (\dot \varphi_{1}^{2} - \dot \varphi_{2})^{2}  &= \frac{\epsilon}{8\pi^{2}}\frac{\varphi_{2}}{k^{2}}(-A\varphi_{2} e^{-\alpha\varphi_{1} (\alpha^{2} \varphi_{1}^{2} - 4\alpha\varphi_{1} + 2 )\dot \varphi_{1}}  \\
    &+ A \varphi_{2} e^{-\alpha\varphi_{1}} (-\alpha^{3} \varphi_{1} ^{2} +6\alpha\varphi_{1} -6\alpha )\dot \varphi_{1} \dot \varphi_{2} + A e^{-\alpha\varphi_{1}} (\alpha^{2} \varphi_{1} ^{2} - 4\alpha \varphi_{1} +2)\dot \varphi_{2} ^{2})
    \end{split}
\end{align}
Repeating the same arguments, we arrive:
\begin{align}
    \begin{split}
        \varphi_{1} = & \varphi_{1}^{0} - \frac{\epsilon}{8\pi^{2}}\frac{m}{6k^{2}}(\varphi_{2}^{0})^{3}[1-3e^{\frac{2s}{\sqrt{k}}} + 2e^{\frac{3s}{\sqrt{k}}}] \\
        & - \frac{\epsilon}{8\pi^{2}}\frac{A}{16k^{2}} e^{-\alpha \tilde{\varphi_{1}}} (-\alpha^{3}\tilde{\varphi_{1}}^{2} + 6\alpha^{2} \tilde{\varphi_{1}} - 6\alpha)(\varphi_{2}^{0})^{4} [1 - 2e^{\frac{2s}{\sqrt{k}}} + e^{\frac{4s}{\sqrt{k}}}]
    \end{split}
\end{align}
\begin{align}
    \begin{split}
        \varphi_{2} = & \varphi_{2}^{0} e^{\frac{s}{\sqrt{k}}} + \frac{\epsilon}{8\pi^{2}}\frac{A}{4k^{5/2}}e^{-\alpha \tilde{\varphi_{1}}} (\alpha^{2} \tilde{\varphi_{1}}^{2} - 4\alpha\tilde{\varphi_{1}}+2)(\varphi_{2}^{0})^{3} e^{\frac{s}{\sqrt{k}}}(-\sqrt{k} - 2s + \sqrt{k} e^{\frac{2s}{\sqrt{k}}})
    \end{split}
\end{align}
Also:
\begin{align}
    \dv{\varphi_{1}}{\varphi_{2}} = -\frac{\epsilon}{8\pi^{2}}\frac{m}{k^{2}}\tilde{\varphi_{2}} (\tilde{\varphi_{2}} - \varphi_{2}^{0}) - \frac{\epsilon}{8\pi^{2}}\frac{A}{4k^{2}}e^{-\alpha\tilde{\varphi_{1}}}(-\alpha^{3} \tilde{\varphi_{1}}^{2} + 6\alpha^{2} \tilde{\varphi_{1}} - 6\alpha)\tilde{\varphi_{2}} (\tilde{\varphi_{2}}^{2} - (\varphi_{2}^{0})^{2}) + \order{\epsilon^{2}}
\end{align}
Then the distance is:
\begin{align}
    \begin{split}
        s =& \sqrt{\tilde{h}_{ab} \dd \varphi_{a} \dd \varphi_{b}} \\
        \sim & \sqrt{k} \ln \abs{\frac{\varphi_{2}}{\varphi_{2}^{0}}} -\frac{\epsilon}{16\pi^{2}}\frac{1}{\sqrt{k}}[ \frac{A}{2k}e^{-\alpha \tilde{\varphi_{1}}}(\alpha^{2}\tilde{\varphi_{1}}^{2} - 4\alpha\tilde{\varphi_{1}}+2) (\varphi_{2}^{2} - (\varphi_{2}^{0})^{2}) -\Lambda^{2} \ln \abs{\frac{\varphi_{2}}{\varphi_{2}^{0}}} ]
    \end{split}
\end{align}
Therefore, the non-trivial constraint which makes the theory satisfy SDC is:
\begin{align}
    \alpha = \frac{2\pm \sqrt{2}}{\tilde{\varphi_{1}}}
\end{align}